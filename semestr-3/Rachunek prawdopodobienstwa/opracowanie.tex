\documentclass[11pt]{article}
\usepackage[utf8]{inputenc}
\usepackage{polski}
\usepackage{graphicx}
\usepackage{array}
\usepackage{paralist}
\usepackage{verbatim}
\usepackage{subfig}
\usepackage{amsmath}
\usepackage{float}
\usepackage{amsthm}
\usepackage{amssymb}
\usepackage{pdfpages}
\usepackage{amsfonts}
\usepackage{tikz}
\usepackage[linguistics]{forest}
\usetikzlibrary{shapes,backgrounds}
\usepackage[margin=1in]{geometry}
\setlength\parindent{0pt}
\theoremstyle{definition}
\newtheorem{zadanie}{Zadanie}
\numberwithin{zadanie}{section}
\renewcommand*{\proofname}{Rozwiązanie}
\maxdeadcycles=1000
\extrafloats{1000}
\title{Algebra i analiza matematyczna}
\author{Igor Nowicki}
\begin{document}
\maketitle
\tableofcontents

\section{Zadania z egzaminów}

\begin{zadanie}
Dla jakich parametrów $m\in\mathbb R$ funkcjonał liniowy

$$f((x_1,x_2), (y_1,y_2)) = x_1y_1+mx_1y_2+mx_2y_1+(3-2m)x_2y_2$$

jest iloczynem skalarnym w $\mathbb R^2$? Dla $m=-1$ wyznaczyć bazę $\mathbb R^2$, w której $f$ ma macierz diagonalną.
\end{zadanie}
\begin{proof}

\end{proof}

\end{document}
